\documentclass[a4,11pt]{article}

% -- Paquetes adicionales
\usepackage{moodle}
\usepackage{listings}

% -- Datos 
\title{Preguntas de probabilidad}
\author{Andrés Merino, Jonathan Ortiz}
\date{Febrero de 2024}


%%%%%%%%%%%%%%%%%%%%%%%%%%%%%%%%%%%%%%%%
\begin{document}

\maketitle

%%%%%%%%%%%%%%%%%%%%%%%%%%%%%%%%%%%%%%%%
\section{Indicaciones}
%%%%%%%%%%%%%%%%%%%%%%%%%%%%%%%%%%%%%%%%

Se plantean preguntas para calcular ciertas probabilidades de una variable aleatoria binomial.


%%%%%%%%%%%%%%%%%%%%%%%%%%%%%%%%%%%%%%%%
%%%%%%%%%%%%%%%%%%%%%%%%%%%%%%%%%%%%%%%%
%% No editar nada de aquí en adelante
%%%%%%%%%%%%%%%%%%%%%%%%%%%%%%%%%%%%%%%%
%%%%%%%%%%%%%%%%%%%%%%%%%%%%%%%%%%%%%%%%

Se utilizó la siguiente pregunta base:
\begin{lstlisting}[breaklines]

\begin{numerical}[tolerance=0.01]%
    % - Indentificador
    {Suma - [[id]]}
    % - Enunciado
    Determine el valor de 
    \[
        [[a]] + [[b]]
    \]
    \item[] [[a+b]]
\end{numerical}

\end{lstlisting}
\noindent
Con los siguientes parámetros:
\begin{itemize}
	\item $a \in \{-2, -1, 0, 1, 2\}$
	\item $b \in \{-2, -1, 1, 2\}$

\end{itemize}
En total, se plantean 2 preguntas.


%%%%%%%%%%%%%%%%%%%%%%%%%%%%%%%%%%%%%%%%
\section{Banco de preguntas}
%%%%%%%%%%%%%%%%%%%%%%%%%%%%%%%%%%%%%%%%

%%%%%%%%%%%%%%%%%%%%%%%%%%%%%%%%%%%%%%%%
\begin{quiz}{Suma}
%%%%%%%%%%%%%%%%%%%%%%%%%%%%%%%%%%%%%%%%

%%%%%%%%%%%%%%%%%%%%%%%%%%%%%%%%%%%%%%%%
%%%%%%%%%%%%%%%%%%%%%%%%%%%%%%%%%%%%%%%%
\begin{numerical}[tolerance=0.01]%
    % - Indentificador
    {Suma - 1}
    % - Enunciado
    Determine el valor de 
    \[
        2 + 1
    \]
    \item[] 3
\end{numerical}

%%%%%%%%%%%%%%%%%%%%%%%%%%%%%%%%%%%%%%%%
\begin{numerical}[tolerance=0.01]%
    % - Indentificador
    {Suma - 2}
    % - Enunciado
    Determine el valor de 
    \[
        -2 + -1
    \]
    \item[] -3
\end{numerical}




\end{quiz}




\end{document}