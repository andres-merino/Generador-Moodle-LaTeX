\documentclass[a4,11pt]{aleph-notas}
% Actualizado en febrero de 2024
% Funciona con TeXLive 2022
% Para obtener solo el pdf, compilar con pdfLaTeX. 
% Para obtener el xml compilar con XeLaTeX.

% -- Paquetes adicionales
\usepackage{aleph-moodle}
\moodleregisternewcommands
% Todos los comandos nuevos deben ir luego del comando anterior
\usepackage{aleph-comandos}

% -- Datos 
\institucion{Proyecto Alephsub0}
% \carrera{}
\asignatura{Probabilidad}
\tema{Banco de preguntas sobre Distribuciones}
\autor{Andrés Merino, Jonathan Ortiz}
\fecha{Octubre 2024}
\logouno[4.5cm]{Logos/LogoAlephsub0-02}
\definecolor{colortext}{HTML}{0030A1}
\definecolor{colordef}{HTML}{00A1DE}
\fuente{montserrat}
% \fuente{mathpazo}

% -- Otros comandos
\newcommand{\Bin}{\text{Bin}}
\newcommand{\Poi}{\text{Poi}}
\newcommand{\Exp}{\text{Exp}}
\newcommand{\Norm}{\text{N}}

\begin{document}

\encabezado

%%%%%%%%%%%%%%%%%%%%%%%%%%%%%%%%%%%%%%%%
\section{Indicaciones}
%%%%%%%%%%%%%%%%%%%%%%%%%%%%%%%%%%%%%%%%


%%%%%%%%%%%%%%%%%%%%%%%%%%%%%%%%%%%%%%%%
%%%%%%%%%%%%%%%%%%%%%%%%%%%%%%%%%%%%%%%%
%% No editar nada de aquí en adelante
%%%%%%%%%%%%%%%%%%%%%%%%%%%%%%%%%%%%%%%%
%%%%%%%%%%%%%%%%%%%%%%%%%%%%%%%%%%%%%%%%

Se utilizó los siguientes parámetros:
\begin{itemize}
	\item $m \in \{3, 4, 5, 6\}$
	\item $u \in \{1, 2, 3\}$
	\item $unaD \in \{True, False\}$

\end{itemize}
En total, se plantean 10 preguntas.


%%%%%%%%%%%%%%%%%%%%%%%%%%%%%%%%%%%%%%%%
\section{Banco de preguntas}
%%%%%%%%%%%%%%%%%%%%%%%%%%%%%%%%%%%%%%%%

%%%%%%%%%%%%%%%%%%%%%%%%%%%%%%%%%%%%%%%%
\begin{quiz}{Normal}
%%%%%%%%%%%%%%%%%%%%%%%%%%%%%%%%%%%%%%%%

%%%%%%%%%%%%%%%%%%%%%%%%%%%%%%%%%%%%%%%%
%%%%%%%%%%%%%%%%%%%%%%%%%%%%%%%%%%%%%%%%
\begin{numerical}[tolerance=0.01]%
    % - Indentificador
    {Normal - 1}
    % - Enunciado
    Sea \( X \sim \Norm(3, 9) \). Calcule \( P(2.25 \leq X \leq 2.64) \).
    \item[] 0.0509
\end{numerical}

%%%%%%%%%%%%%%%%%%%%%%%%%%%%%%%%%%%%%%%%
\begin{numerical}[tolerance=0.01]%
    % - Indentificador
    {Normal - 2}
    % - Enunciado
    Sea \( X \sim \Norm(3, 4) \). Calcule \( P(0.05 \leq X \leq 5.3) \).
    \item[] 0.8048
\end{numerical}

%%%%%%%%%%%%%%%%%%%%%%%%%%%%%%%%%%%%%%%%
\begin{numerical}[tolerance=0.01]%
    % - Indentificador
    {Normal - 3}
    % - Enunciado
    Sea \( X \sim \Norm(6, 4) \). Calcule \( P(8.1 \leq X \leq 9.45) \).
    \item[] 0.1046
\end{numerical}

%%%%%%%%%%%%%%%%%%%%%%%%%%%%%%%%%%%%%%%%
\begin{numerical}[tolerance=0.01]%
    % - Indentificador
    {Normal - 4}
    % - Enunciado
    Sea \( X \sim \Norm(5, 4) \). Calcule \( P(X \leq 1.81) \).
    \item[] 0.0554
\end{numerical}

%%%%%%%%%%%%%%%%%%%%%%%%%%%%%%%%%%%%%%%%
\begin{numerical}[tolerance=0.01]%
    % - Indentificador
    {Normal - 5}
    % - Enunciado
    Sea \( X \sim \Norm(4, 9) \). Calcule \( P(X \leq 5.36) \).
    \item[] 0.6748
\end{numerical}

%%%%%%%%%%%%%%%%%%%%%%%%%%%%%%%%%%%%%%%%
\begin{numerical}[tolerance=0.01]%
    % - Indentificador
    {Normal - 6}
    % - Enunciado
    Sea \( X \sim \Norm(4, 1) \). Calcule \( P(5.07 \leq X \leq 6.24) \).
    \item[] 0.1298
\end{numerical}

%%%%%%%%%%%%%%%%%%%%%%%%%%%%%%%%%%%%%%%%
\begin{numerical}[tolerance=0.01]%
    % - Indentificador
    {Normal - 7}
    % - Enunciado
    Sea \( X \sim \Norm(4, 1) \). Calcule \( P(X \leq 2.3) \).
    \item[] 0.0446
\end{numerical}

%%%%%%%%%%%%%%%%%%%%%%%%%%%%%%%%%%%%%%%%
\begin{numerical}[tolerance=0.01]%
    % - Indentificador
    {Normal - 8}
    % - Enunciado
    Sea \( X \sim \Norm(5, 4) \). Calcule \( P(1.65 \leq X \leq 10.71) \).
    \item[] 0.9509
\end{numerical}

%%%%%%%%%%%%%%%%%%%%%%%%%%%%%%%%%%%%%%%%
\begin{numerical}[tolerance=0.01]%
    % - Indentificador
    {Normal - 9}
    % - Enunciado
    Sea \( X \sim \Norm(5, 9) \). Calcule \( P(X \geq 5.68) \).
    \item[] 0.4103
\end{numerical}

%%%%%%%%%%%%%%%%%%%%%%%%%%%%%%%%%%%%%%%%
\begin{numerical}[tolerance=0.01]%
    % - Indentificador
    {Normal - 10}
    % - Enunciado
    Sea \( X \sim \Norm(3, 4) \). Calcule \( P(X \leq 4.2) \).
    \item[] 0.7257
\end{numerical}




\end{quiz}




\end{document}