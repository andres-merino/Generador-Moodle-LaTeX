\documentclass[a4,11pt]{article}

% -- Paquetes adicionales
\usepackage{moodle}
\usepackage{listings}

% -- Datos 
\title{Preguntas de probabilidad}
\author{Andrés Merino, Jonathan Ortiz}
\date{Febrero de 2024}


%%%%%%%%%%%%%%%%%%%%%%%%%%%%%%%%%%%%%%%%
\begin{document}

\maketitle

%%%%%%%%%%%%%%%%%%%%%%%%%%%%%%%%%%%%%%%%
\section{Indicaciones}
%%%%%%%%%%%%%%%%%%%%%%%%%%%%%%%%%%%%%%%%

Se plantean preguntas para calcular ciertas probabilidades de una variable aleatoria binomial.


%%%%%%%%%%%%%%%%%%%%%%%%%%%%%%%%%%%%%%%%
%%%%%%%%%%%%%%%%%%%%%%%%%%%%%%%%%%%%%%%%
%% No editar nada de aquí en adelante
%%%%%%%%%%%%%%%%%%%%%%%%%%%%%%%%%%%%%%%%
%%%%%%%%%%%%%%%%%%%%%%%%%%%%%%%%%%%%%%%%

Se utilizó la siguiente pregunta base:
\begin{lstlisting}[breaklines]
{{Enunciado}}
\end{lstlisting}
\noindent
Con los siguientes parámetros:
\begin{itemize}
{{Parámetros}}
\end{itemize}
En total, se plantean {{Número de preguntas}} preguntas.


%%%%%%%%%%%%%%%%%%%%%%%%%%%%%%%%%%%%%%%%
\section{Banco de preguntas}
%%%%%%%%%%%%%%%%%%%%%%%%%%%%%%%%%%%%%%%%

%%%%%%%%%%%%%%%%%%%%%%%%%%%%%%%%%%%%%%%%
\begin{quiz}{{{Cuestionario}}}
%%%%%%%%%%%%%%%%%%%%%%%%%%%%%%%%%%%%%%%%

%%%%%%%%%%%%%%%%%%%%%%%%%%%%%%%%%%%%%%%%
{{QUIZ}}


\end{quiz}




\end{document}